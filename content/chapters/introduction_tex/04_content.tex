\label{chap1:sec4-thesis-content}

\section{Thesis Contents}
This introductory chapter has laid the foundational groundwork for the thesis, establishing the critical context for data annotation, identifying the significant research gap stemming from limitations in existing platforms, and clearly articulating the proposed solution through specific objectives and its defined scope. The subsequent chapters will delve into the detailed design, implementation, and rigorous evaluation of the proposed web-based data annotation platform.

\subsection*{Chapter 2. Literature Review}
  This chapter will provide a comprehensive and critical review of existing academic and industry literature pertinent to data annotation platforms, advanced workflow management systems, modern access control mechanisms, and the principles of auditability in complex software systems. It will analyze current approaches, identify their strengths and weaknesses, and precisely delineate the specific gaps that this thesis aims to address.

\subsection*{Chapter 3. System Design and Architecture}
 This chapter will detail the architectural design of the proposed web-based data annotation platform. It will elaborate on the chosen architectural patterns, the interconnections between its backend, middleware, and frontend components, and the specific design principles underpinning the dynamic, graph-based workflow engine, the fine-grained Attribute-Based Access Control (ABAC) system, and the comprehensive audit trail mechanism.

\subsection*{Chapter 4. Implementation and Evaluation}
 This chapter will describe the detailed implementation of the proposed data annotation platform and present comprehensive evaluation results. It will outline the specific technologies, programming languages, frameworks, and methodologies employed in developing each core component. The chapter includes three evaluation scenarios: linear workflows, conditional workflows with routing logic, and Model-in-the-Loop integration. Furthermore, it will assess the platform's performance, reliability, and scalability through quantitative metrics and real-world use cases.

\subsection*{Chapter 5. Conclusion and Future Work}
 This final chapter will synthesize the key findings and contributions of the thesis to the field of data annotation and machine learning infrastructure. It will also critically discuss the limitations of the current work and propose promising directions for future research and development, suggesting avenues for extending the platform's capabilities and applicability.