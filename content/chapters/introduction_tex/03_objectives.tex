\section{Objective}
\label{sec:chapter-1-objectives}

This thesis aims to address shortcomings in existing data annotation platforms by designing and implementing a system with enhanced flexibility, security, extensibility, and auditability. The specific objectives are meticulously crafted to contribute directly to a more efficienta and reliable data annotation process for machine learning.

\subsection*{Objective 1. To design and implement a data annotation platform with a dynamic, graph-based workflow engine.}
This objective seeks to overcome the inherent rigidity of current annotation workflows by introducing a dynamic, graph-based engine. Conceptualized as a Directed Acyclic Graph (DAG), this engine will facilitate the flexible definition and execution of complex, multi-stage annotation pipelines, adapting to evolving requirements in real-time. Such an approach offers unparalleled flexibility, scalability, and efficiency by streamlining processes, reducing bottlenecks, and enabling real-time adjustments to rules and conditions. A visual, low-code pipeline builder will democratize the creation and management of sophisticated data preparation strategies. Furthermore, the dynamic workflow engine will provide operational intelligence through real-time insights and analytics, enabling identification of bottlenecks and optimization of resource allocation. This directly contributes to the agility necessary for iterative AI development, accelerating the machine learning lifecycle by allowing rapid experimentation with diverse annotation strategies.

\subsection*{Objective 2. To develop a fine-grained, Attribute-Based Access Control (ABAC) system for secure and contextual permissions.}
This objective directly addresses the limitations of inflexible access control by implementing an Attribute-Based Access Control (ABAC) system. ABAC determines permissions based on a dynamic combination of user, resource, and environmental attributes (e.g., job role, data classification, access location). Access decisions are made in real-time by evaluating these attributes against predefined policies. The benefits of ABAC are substantial: it enhances security by enforcing granular, context-aware controls; aids in compliance with stringent regulations (e.g., GDPR, HIPAA, NIST 800-53); scales effortlessly in rapidly growing enterprises and cloud environments; and reduces administrative burden by centralizing permission management. By dynamically evaluating a rich set of attributes, ABAC enables proactive risk mitigation, ensuring that access is granted or denied based on the contextual risk of an access attempt. This represents a fundamental shift from static, identity-based security to a context-aware approach, crucial for handling sensitive data and enabling secure collaboration.

\subsection*{Objective 3. To ensure the platform is extensible for integration with external data sources and machine learning models}
This objective underscores the platform's architectural design for seamless integration with diverse external data sources and various machine learning models. Extensibility will be achieved through comprehensive APIs facilitating automated imports, task creation, and exports, alongside robust support for cloud storage solutions. Crucially, the platform will support Man-in-the-Loop (MITL), also known as Human-in-the-Loop (HITL), systems. This involves leveraging AI-assisted labeling (e.g., pre-annotation using models like SAM or Grounding DINO, active learning) for initial suggestions, combined with robust human validation for quality control, bias mitigation, and handling complex edge cases. For specialized domains, such as medical imaging, extensibility will be demonstrated through integration with established tools like the OHIF Viewer for visualization, MONAI Label for AI-assisted annotation tailored for medical images, and Orthanc as a PACS server for data management. This symbiotic relationship between AI and human expertise will lead to continuously improving data quality and model performance, while the extensive integration capabilities position the platform as an integral component within broader MLOps pipelines.

\subsection*{Objective 4. To create a reliable system with comprehensive audit trails for all actions and state transitions.}
This objective aims to ensure the platform's reliability, traceability, and accountability by implementing robust audit logging and audit trail functionalities. Audit logging systematically records all significant activities and changes, capturing detailed information about user actions and system events. Audit trails then connect these discrete events into a detailed, narrative-driven timeline, illustrating the lineage and transformation of data. The importance of comprehensive audit trails is multifaceted: they provide accountability and traceability by documenting who performed each action, when, and why; they are vital for demonstrating compliance with legal and regulatory frameworks (e.g., GDPR, HIPAA, FDA 21 CFR Part 11); they enable early detection of unauthorized access or suspicious activities; and they assist in debugging and root cause analysis. Furthermore, transparency into data sources and transformations, facilitated by audit trails, fosters increased trust among stakeholders, helps identify and mitigate biases in datasets, and enhances the interpretability and explainability of AI models by meticulously tracking data lineage.