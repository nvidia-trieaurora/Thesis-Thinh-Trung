\chapter{Introduction}

\section{Problem Overview}

In the context of increasing demand for high-quality annotated medical imaging datasets to support research and the development of AI-assisted diagnostic applications, the current data annotation process at many hospitals in Vietnam still faces several practical challenges. Specifically, the majority of annotation tasks remain heavily reliant on manual efforts from radiologists, resulting in significant time consumption and resource constraints, especially for complex cases involving detailed analysis of multi-slice CT and MRI scans. Furthermore, there is a lack of a structured and coherent workflow management system, with limited task assignment mechanisms and progress monitoring capabilities. This leads to fragmented coordination among annotation team members and limits the efficiency of human resources. Additionally, communication and professional feedback between radiologists and annotation staff are often not sufficiently facilitated, posing difficulties in handling complex clinical cases.

Moreover, the potential of modern AI-powered tools remains underutilized in practical workflows, as most annotation tasks continue to be performed manually, even in scenarios where automation could be effectively applied.

These limitations highlight the pressing need for the development of an intelligent annotation system that provides a well-defined and streamlined workflow, enables clear task delegation, facilitates effective collaboration among clinical professionals, and integrates advanced AI tools to reduce annotation time while enhancing the quality of annotated medical data.

\section{Objectives and Contributions}

\subsection{Primary Objectives}

This thesis aims to develop a comprehensive intelligent annotation system for medical imaging data that addresses the challenges outlined above. The primary objectives include:

\textbf{Objective 1: Develop an Integrated Annotation Platform}
Create a unified platform that combines state-of-the-art medical image viewing capabilities with AI-assisted annotation tools, providing an intuitive interface for medical professionals to efficiently annotate complex medical imaging data.

\textbf{Objective 2: Implement AI-Assisted Segmentation}
Integrate advanced AI models, including foundation models like SAM and interactive segmentation tools, to provide intelligent assistance in identifying and segmenting anatomical structures and pathological findings.

\textbf{Objective 3: Design Flexible Workflow Management}
Implement a robust workflow management system that supports various annotation workflows, task assignment, quality control processes, and collaborative annotation scenarios.

\textbf{Objective 4: Ensure System Scalability and Integration}
Design the system architecture to support scalability and integration with existing medical imaging infrastructure, including PACS systems and DICOM-compliant storage solutions.

\subsection{Specific Contributions}

This research makes the following specific contributions to the field of medical data annotation:

\textbf{System Development Contributions:}
\begin{itemize}
    \item Development of an integrated annotation platform that combines state-of-the-art medical image viewing with AI-assisted annotation tools
    \item Implementation of a structured workflow management system with clear task assignment mechanisms, progress monitoring capabilities, and efficient task delegation to optimize human resource utilization
    \item Creation of real-time collaborative annotation features facilitating effective communication and professional feedback between clinical staff, including professional communication tools supporting complex clinical case discussions
    \item Development and integration of advanced AI tools including customized foundation models to reduce manual annotation workload while maintaining quality
    \item Creation of a deployment-ready system addressing practical challenges in Vietnamese healthcare contexts
\end{itemize}

\textbf{Research Contributions:}
\begin{itemize}
    \item Analysis of workflow management challenges in medical annotation processes
    \item Evaluation of intelligent annotation system effectiveness in improving coordination and reducing manual workload
    \item Research on integrating AI-assisted tools into existing hospital annotation workflows
    \item Framework for structured collaboration between radiologists and annotation staff in clinical environments
\end{itemize}

\subsection{Scope and Limitations}

\textbf{Scope of Research:}
This research focuses on developing an annotation system primarily for CT and MRI imaging modalities using DICOM format. The system targets common annotation tasks including organ segmentation, lesion identification, and anatomical structure labeling. The implementation concentrates on creating a functional prototype suitable for deployment in Vietnamese healthcare settings.

\textbf{Research Limitations:}
\begin{itemize}
    \item The system is currently validated and tested specifically at Thong Nhat Hospital, with plans for future expansion to additional healthcare institutions
    \item AI model performance is dependent on the quality and diversity of training data available from the current testing environment
    \item Clinical validation is limited to the workflows and practices observed at the primary testing site
    \item Integration testing focuses on the existing infrastructure at Thong Nhat Hospital and may require adaptation for different institutional setups
\end{itemize}

\section{Thesis Organization}

This thesis is organized into six chapters that systematically address the research objectives and present the developed solution:

\textbf{Chapter 1: Introduction} provides an overview of the research problem, motivation, objectives, and contributions. It establishes the context for medical data annotation challenges and outlines the approach taken in this research.

\textbf{Chapter 2: Literature Review and Related Work} examines existing research in medical data annotation techniques, current annotation support systems, AI applications in medical imaging, and workflow management solutions. This chapter identifies gaps in current approaches and establishes the theoretical foundation for the proposed solution.

\textbf{Chapter 3: Problem Analysis and Proposed Solution} presents a detailed analysis of medical annotation challenges and introduces the proposed intelligent annotation system. It defines system requirements, presents the overall solution architecture, and explains the integration approach.

\textbf{Chapter 4: Detailed System Design} describes the technical implementation of the proposed system, including detailed architecture design, individual module specifications, technology stack selection, and integration strategies. This chapter provides the technical foundation for system implementation.

\textbf{Chapter 5: Implementation and Evaluation} presents the practical implementation of the system, deployment considerations, performance evaluation results, and case studies demonstrating the system's effectiveness. It includes quantitative and qualitative analysis of the system's impact on annotation workflows.

\textbf{Chapter 6: Conclusion and Future Work} summarizes the research contributions, discusses limitations and challenges encountered, and outlines directions for future research and development. It concludes with an assessment of the system's potential impact on medical AI annotation practices.

The thesis also includes technical appendices providing detailed system specifications, user guides, and additional implementation details to support future development and deployment efforts. 