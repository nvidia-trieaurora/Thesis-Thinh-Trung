\chapter{Literature Review and Related Work}

\section{Medical Data Annotation Techniques}

\subsection{Traditional Medical Annotation Methods}

Medical image annotation has traditionally relied on manual processes performed by expert radiologists and medical professionals. These conventional approaches have formed the foundation of medical imaging interpretation for decades and continue to be the gold standard for accuracy and reliability.

\textbf{Manual Annotation Workflows:} Traditional medical annotation involves radiologists examining medical images slice by slice, identifying anatomical structures, pathological findings, and regions of interest. This process typically includes contouring organs, marking lesions, measuring distances, and providing descriptive annotations. The workflow generally follows a systematic approach where experts review images in multiple planes (axial, sagittal, and coronal) to ensure comprehensive coverage and accuracy.

\textbf{Expert-Based Quality Control:} Conventional annotation workflows incorporate multiple levels of review, often involving senior radiologists validating annotations made by junior colleagues or residents. This hierarchical approach ensures quality but significantly increases the time and cost associated with annotation processes.

\textbf{Standardization Efforts:} Medical communities have developed various standardization protocols to ensure consistency in annotation practices. These include anatomical atlases, standardized terminology systems such as RadLex, and institutional guidelines that aim to reduce inter-observer variability.

\subsection{DICOM Standards and Medical Imaging Infrastructure}

The Digital Imaging and Communications in Medicine (DICOM) standard plays a crucial role in medical imaging annotation by providing a comprehensive framework for storing, transmitting, and managing medical images and associated metadata.

\textbf{DICOM Structure and Annotation Support:} DICOM supports various types of annotations through specialized objects such as DICOM Structured Reports (SR), DICOM Segmentation Objects, and DICOM Presentation States. These standardized formats enable interoperability between different medical imaging systems and ensure that annotations can be consistently interpreted across various platforms and institutions.

\textbf{DICOM Segmentation Objects:} Modern DICOM standards include support for storing segmentation results as DICOM objects, enabling the preservation of automated and manual segmentation results within the standard medical imaging workflow. This capability is essential for integrating AI-assisted annotation tools with existing medical imaging infrastructure.

\textbf{Structured Reporting:} DICOM Structured Reports provide a standardized mechanism for storing complex annotation data, including measurements, observations, and diagnostic conclusions. This format enables machine-readable storage of annotation results while maintaining human readability and clinical relevance.

\subsection{Challenges in Medical Annotation}

Medical image annotation faces several significant challenges that limit its efficiency and scalability:

\textbf{Complexity and Expertise Requirements:} Medical images contain complex anatomical structures with subtle variations that require extensive medical training to interpret accurately. The need for specialized knowledge creates a bottleneck in annotation workflows, as qualified experts are limited and expensive.

\textbf{Inter-observer Variability:} Studies have consistently shown significant variability between different experts annotating the same medical images. This variability can impact the quality of datasets used for training AI models and affect the reproducibility of research results.

\textbf{Time and Resource Constraints:} Manual annotation is extremely time-consuming, with complex cases requiring hours of expert time. This limitation severely restricts the scale at which annotated medical datasets can be created, hindering the development of large-scale AI applications in healthcare.

\section{Current Annotation Support Systems}

\subsection{OHIF Viewer Platform}

The Open Health Imaging Foundation (OHIF) Viewer represents one of the most successful open-source platforms for web-based medical image viewing and annotation.

\textbf{Architecture and Capabilities:} OHIF Viewer is built as a progressive web application using modern JavaScript frameworks, enabling deployment across various platforms without requiring specialized software installation. The platform supports multi-modal medical imaging data and provides comprehensive tools for image visualization, measurement, and annotation.

\textbf{DICOM Compliance and Integration:} OHIF Viewer provides extensive support for DICOM standards, including DICOMweb protocols for data retrieval and storage. This compliance enables seamless integration with existing Picture Archiving and Communication Systems (PACS) and medical imaging infrastructure commonly found in healthcare institutions.

\textbf{Extensibility and Customization:} The platform's modular architecture allows for extensive customization and extension through a plugin system. This flexibility has made OHIF Viewer popular for research applications and custom medical imaging solutions.

\textbf{Limitations and Gaps:} While OHIF Viewer provides excellent visualization capabilities, it lacks built-in AI assistance features and sophisticated workflow management tools. The platform requires significant customization to support complex annotation workflows and collaborative features needed for large-scale annotation projects.

\subsection{MONAI Label Framework}

MONAI Label represents a significant advancement in AI-assisted medical image annotation, providing a comprehensive framework for integrating machine learning models into annotation workflows.

\textbf{AI-Assisted Annotation Capabilities:} MONAI Label incorporates state-of-the-art deep learning models for automatic and interactive segmentation of medical images. The framework supports various AI models, including foundation models like VISTA3D, interactive segmentation tools like DeepEdit, and specialized models for specific anatomical structures.

\textbf{Interactive Segmentation Approach:} The framework implements interactive segmentation techniques that allow users to guide AI models through point clicks, bounding boxes, and partial annotations. This approach combines the efficiency of automated segmentation with the precision of expert guidance.

\textbf{Model Training and Adaptation:} MONAI Label includes capabilities for continuous learning and model adaptation based on user corrections and feedback. This feature enables the system to improve over time and adapt to specific institutional requirements and annotation styles.

\textbf{Integration Challenges:} While MONAI Label provides powerful AI capabilities, its integration with existing medical imaging workflows and PACS systems requires significant technical expertise. The framework primarily focuses on the AI components and lacks comprehensive workflow management and user interface components needed for complete annotation solutions.

\subsection{Encord and Commercial Workflow Management}

Encord represents the commercial state-of-the-art in annotation workflow management, providing comprehensive tools for managing large-scale annotation projects.

\textbf{Workflow Management Features:} Encord provides sophisticated workflow management capabilities including task assignment, progress tracking, quality control processes, and collaborative annotation features. The platform supports complex annotation pipelines with multiple review stages and automated quality assurance mechanisms.

\textbf{Quality Control and Validation:} The platform includes comprehensive quality control features such as inter-annotator agreement metrics, automated quality checks, and hierarchical review processes. These features help ensure annotation consistency and accuracy across large teams and projects.

\textbf{Collaboration and Team Management:} Encord supports advanced collaboration features including real-time annotation sharing, team management tools, and communication features that facilitate coordination among distributed annotation teams.

\textbf{Limitations for Medical Applications:} While Encord provides excellent general-purpose annotation management capabilities, it lacks specific features required for medical imaging applications such as DICOM integration, medical-specific annotation tools, and compliance with healthcare regulatory requirements.

\subsection{Comparative Analysis of Current Systems}

\begin{table}[htbp]
\centering
\caption{Comparison of Current Annotation Support Systems}
\label{tab:system-comparison}
\begin{tabular}{|p{2.5cm}|p{3cm}|p{3cm}|p{3cm}|}
\hline
\textbf{Feature} & \textbf{OHIF Viewer} & \textbf{MONAI Label} & \textbf{Encord} \\
\hline
Medical Imaging Support & Excellent & Good & Limited \\
\hline
AI Assistance & None & Excellent & Limited \\
\hline
Workflow Management & Basic & None & Excellent \\
\hline
DICOM Integration & Excellent & Good & None \\
\hline
Collaborative Features & Limited & None & Excellent \\
\hline
\end{tabular}
\end{table}

The analysis reveals that while each system excels in specific areas, no single platform provides comprehensive coverage of all requirements for intelligent medical annotation workflows.

\section{AI in Medical Annotation}

\subsection{Interactive Segmentation Techniques}

Interactive segmentation represents a paradigm shift in medical image annotation, combining the efficiency of automated algorithms with the precision of expert guidance.

\textbf{Point-Based Interaction Methods:} Modern interactive segmentation techniques allow users to guide segmentation algorithms through simple point clicks indicating foreground and background regions. These methods, such as those implemented in DeepEdit, enable rapid refinement of automated segmentation results with minimal user effort.

\textbf{Scribble-Based Approaches:} Advanced interactive methods support scribble-based input where users can draw rough outlines or corrections that guide the segmentation algorithm. This approach is particularly effective for complex anatomical structures where point-based methods may be insufficient.

\textbf{Bounding Box and Geometric Constraints:} Some interactive segmentation methods utilize geometric constraints such as bounding boxes, ellipses, or other geometric primitives to guide the segmentation process. These methods are particularly useful for well-defined anatomical structures with predictable shapes.

\subsection{Model-in-the-Loop (MITL) Approaches}

Model-in-the-Loop approaches represent an advanced paradigm where AI models are continuously integrated into the annotation workflow, learning and adapting from user interactions.

\textbf{Continuous Learning Framework:} MITL systems continuously update their models based on user corrections and feedback, enabling progressive improvement in annotation quality and efficiency. This approach is particularly valuable in medical applications where institutional variations and specific requirements need to be accommodated.

\textbf{Active Learning Integration:} Advanced MITL systems incorporate active learning strategies that intelligently select the most informative cases for human annotation, maximizing the learning value of expert effort and accelerating model improvement.

\textbf{Uncertainty Quantification:} Modern MITL approaches include uncertainty quantification mechanisms that help identify cases where automated annotations may be unreliable, directing expert attention to areas where human input is most valuable.

\subsection{Foundation Models in Medical Imaging}

Recent advances in foundation models have revolutionized the landscape of medical image analysis and annotation.

\textbf{VISTA3D and Medical Foundation Models:} VISTA3D represents a breakthrough in medical imaging AI, providing a foundation model capable of segmenting over 127 anatomical structures across various medical imaging modalities. This capability significantly reduces the need for specialized model training for each anatomical structure or imaging protocol.

\textbf{Transfer Learning and Adaptation:} Foundation models enable efficient adaptation to new medical imaging tasks through transfer learning approaches. This capability allows the rapid deployment of AI assistance for new anatomical regions or pathological conditions without requiring extensive training data.

\textbf{MONAI-Based AI Annotation Approaches:} Modern AI annotation systems leverage multiple interaction paradigms implemented in MONAI framework. \textit{Auto-Segmentation} provides fully automated initial segmentation using pre-trained foundation models, enabling rapid baseline annotations across multiple anatomical structures. \textit{Class Prompts} allow users to specify target anatomical structures or pathological findings through text-based or categorical inputs, guiding the AI model to focus on specific regions of interest. \textit{Point Prompts} enable interactive refinement through user-provided click points, where positive clicks indicate target regions and negative clicks exclude unwanted areas, allowing for precise manual guidance of the automated segmentation process. This multi-modal approach combines the efficiency of automated processing with the precision of expert-guided interaction.

\section{Workflow Management in Healthcare}

\subsection{Task Assignment and Tracking Systems}

Effective workflow management in medical annotation requires sophisticated task assignment and tracking capabilities that account for the specialized nature of medical expertise and the critical importance of annotation quality.

\textbf{Expertise-Based Assignment:} Medical annotation workflows must consider the specific expertise and qualifications of annotators when assigning tasks. Different medical specialties may be required for different types of imaging studies or anatomical regions.

\textbf{Workload Balancing:} Effective systems must balance workload across available annotators while considering factors such as case complexity, annotator experience, and quality requirements. This balancing is crucial for maintaining consistent productivity and quality across large annotation projects.

\textbf{Progress Monitoring:} Comprehensive tracking systems provide real-time visibility into annotation progress, quality metrics, and potential bottlenecks. This information enables project managers to make informed decisions about resource allocation and timeline adjustments.

\subsection{Quality Control Processes}

Quality control in medical annotation requires specialized processes that ensure both accuracy and consistency while maintaining reasonable throughput.

\textbf{Multi-Stage Review Processes:} Medical annotation workflows typically incorporate multiple review stages, including initial annotation, expert review, and final validation. Each stage may involve different levels of expertise and different quality criteria.

\textbf{Automated Quality Checks:} Modern systems incorporate automated quality checks that can identify potential errors or inconsistencies in annotations. These checks may include geometric consistency validation, anatomical plausibility assessment, and comparison with reference standards.

\textbf{Inter-Annotator Agreement Metrics:} Quality control systems must provide mechanisms for measuring and tracking inter-annotator agreement, enabling identification of cases requiring additional review or consensus building.

\subsection{Collaboration Tools and Communication}

Effective collaboration tools are essential for managing distributed annotation teams and facilitating communication between annotators, reviewers, and project managers.

\textbf{Real-Time Collaboration Features:} Modern annotation platforms support real-time collaboration features that enable multiple experts to work simultaneously on the same case or to provide immediate consultation and feedback.

\textbf{Communication and Annotation History:} Comprehensive systems maintain detailed histories of annotation activities, including comments, revisions, and communication between team members. This information is crucial for maintaining annotation quality and resolving disagreements.

\textbf{Knowledge Sharing Platforms:} Advanced collaboration systems include knowledge sharing features such as annotation guidelines, reference cases, and best practice documentation that help maintain consistency across large annotation teams.

\section{Research Gap Analysis}

\subsection{Current System Limitations}

Analysis of existing annotation systems reveals key practical limitations:

\textbf{System Fragmentation:} Healthcare institutions typically use separate tools for image viewing (OHIF), AI assistance (MONAI Label), and workflow management (manual processes), requiring complex integration and causing inefficiencies.

\textbf{Workflow Management Gaps:} Current systems lack structured task assignment, progress monitoring, and quality control mechanisms essential for hospital environments. MONAI Label provides AI capabilities but no workflow coordination, while OHIF offers excellent viewing but minimal collaboration features.

\textbf{Integration Complexity:} Implementing AI-assisted annotation in hospital settings requires significant technical expertise to connect PACS systems, configure AI models, and maintain consistent data formats across different platforms.

\subsection{Research Opportunities}

These limitations create specific opportunities addressed in this research:

\textbf{Integrated Platform:} Combining OHIF's medical imaging capabilities with MONAI's AI assistance and adding comprehensive workflow management into a unified system designed for Vietnamese healthcare contexts.

\textbf{Practical Deployment:} Developing a solution that can be readily deployed at institutions like Thong Nhat Hospital without requiring extensive technical infrastructure or specialized IT support.

\textbf{Collaborative Workflows:} Implementing real-time collaboration features, structured review processes, and communication tools specifically designed for medical annotation teams working on complex clinical cases. 