\begin{EnAbstract}
Medical imaging annotation remains a critical bottleneck in healthcare AI development, requiring extensive manual effort from medical experts and limiting the scalability of AI applications in clinical settings. This thesis presents the design, implementation, and evaluation of an intelligent medical annotation system that integrates AI assistance with clinical workflows to address these challenges.

The proposed solution consists of a four-component microservices architecture: (1) a React-based workflow management platform using Supabase for real-time collaboration and task orchestration; (2) an AI-enhanced annotation interface built on OHIF Viewer 3.x with integrated AI assistance capabilities; (3) a MONAI Label-based AI engine providing foundation model support including VISTA3D for multi-organ segmentation; and (4) an Orthanc PACS system for DICOM-compliant medical data management. This integrated approach enables seamless coordination between AI capabilities and medical expertise while maintaining compatibility with existing healthcare infrastructure.

The system implements novel AI-human collaboration mechanisms including interactive segmentation refinement, uncertainty visualization, and adaptive learning based on user corrections. A sophisticated workflow orchestration engine supports various annotation paradigms from simple single-annotator tasks to complex multi-stage collaborative workflows with quality control and consensus building.

The system design enables seamless integration with existing PACS infrastructure through standard DICOM protocols and provides a scalable foundation for AI-assisted medical annotation in healthcare environments. The platform demonstrates the feasibility of combining multiple specialized medical imaging tools into a unified, user-friendly interface that can adapt to various institutional requirements and clinical workflows.

The key contributions include a novel microservices integration architecture for medical imaging tools, an innovative AI-human collaboration framework optimized for medical annotation workflows, and a practical implementation that demonstrates the viability of AI-assisted annotation in clinical settings. This work provides a foundation for scalable medical annotation systems and contributes to advancing healthcare AI adoption in resource-constrained environments.

\end{EnAbstract}