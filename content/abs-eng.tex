\begin{EnAbstract}

In the rapidly developing landscape of healthcare AI, medical image annotation remains a major challenge, requiring substantial time from experts and directly impacting the scalability of AI systems in hospitals. Drawing from practical difficulties encountered in deployment projects, this thesis presents the design, construction, and evaluation of an intelligent annotation system that integrates AI into hospital workflows to reduce manual workload and enhance data labeling efficiency.

The system is developed on a microservices architecture with four main components: (1) a React-based workflow management platform using Supabase for real-time collaboration, including detailed permissions and multi-stage processes with comment and feedback features to facilitate interaction between doctors and annotators; (2) an AI-integrated annotation interface on OHIF Viewer 3.x for visualizing medical data and providing effective support tools; (3) an AI module using MONAI Label, supporting foundation models for multi-organ segmentation; (4) the Orthanc PACS system for storing and managing medical imaging data compliant with DICOM standards.

The system's distinction lies in combining AI with human expertise through mechanisms such as interactive refinement to enhance accuracy on project-specific data, along with UNET model optimization to reduce model size while maintaining clinically suitable accuracy. The system also supports diverse working scenarios from individual to group settings, incorporating quality check steps to ensure reliability.

The design emphasizes easy integration with PACS systems via DICOM standards, aiming for flexibility across various clinical situations. The system has been tested in practice with input from doctors, capturing real feedback to further refine the solution.

The thesis's main contributions center on three aspects: (1) improving workflows through detailed permissions and multi-stage processes with doctor involvement; (2) optimizing AI models to balance operational efficiency and required accuracy; (3) conducting real evaluations with experts to ensure feasibility for hospital implementation. This research aims to contribute to more efficient annotation systems, supporting AI development in healthcare, particularly in resource-limited environments.

\end{EnAbstract}